\newpage
\section*{Определения, обозначения и сокращения}
\addcontentsline{toc}{section}{Определения, обозначения и сокращения}

В расчетно-пояснительной записке использованы следующие Определения, обозначения и сокращения:

API - интерфейс программирования приложений. набор готовых классов, процедур, функций, структур и констант, предоставляемых приложением (библиотекой, сервисом) для использования во внешних программных продуктах.

Android -  операционная система для смартфонов, планшетных компьютеров, электронных книг, цифровых проигрывателей, наручных часов, игровых приставок, нетбуков, смартбуков, очков Google и других устройств. Основана на ядре Linux и собственной реализации Java от Google.

C++ - компилируемый статически типизированный язык программирования общего назначения.

GPS - Global Positioning System.  Спутниковая система навигации, обеспечивающая измерение расстояния, времени и определяющая местоположениe во всемирной системе координат. 

Java -  объектно-ориентированный язык программирования, разработанный компанией Sun Microsystems.

MATLAB -  пакет прикладных программ для решения задач технических вычислений и одноимённый язык программирования, используемый в этом пакете.

SLAM - Simultaneous Location and Mapping. Метод, используемый роботами и автономными транспортными средствами для построения карты в неизвестном пространстве или для обновления карты в заранее известном пространстве с одновременным контролем текущего местоположения и пройденного пути.

ИИУ - инерционное измерительное устройство. Устройство включающее в себя датчики линейного ускорения (акселерометры) и угловой скорости (гироскопы или пары акселерометров, измеряющих центробежное ускорение).  
 
СЛАУ - система линейных алгебраических уравнений.
Дрейф нуля - самопроизвольное изменение выходного сигнала со временем при неизменном или даже отсутствующем входном сигнале.

ГЛОНАСС - глобальная навигационная спутниковая система. советская и российская спутниковая система навигации, разработана по заказу Министерства обороны СССР.

Фреймворк - структура программной системы; программное обеспечение, облегчающее разработку и объединение разных компонентов большого программного проекта.

Одомтерия - использование данных о движении приводов, для оценки перемещения. Так же термин применяется для любого метода оценки перемещения.