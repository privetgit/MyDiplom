\newpage

\section{Настройка рабочих параметров сетевой ОС}

Согласно ТЗ, в качестве сетевой ОС выбрана ОС MS Windows 7.\par\bigskip

Когда речь идет о подключении компьютера с Windows 7 к локальной сети, процесс настройки механизмов очень похож на аналогичный процесс в более ранних версиях ОС MS Windows. Имеются также и определенные отличия, например в использовании определенного типа сетевого расположения.\par\bigskip

Выбор сетевого расположения влияет на защиту операционной системы от возможных воздействий локальной сети, поэтому, если правильно подобрать тип расположения, это позволит получить большую защиту.\par\bigskip

Однако данный параметр критичен только в случае, когда компьютер функционирует в составе одноранговой сети и подключен к рабочей или домашней группе. Если же компьютер входит в состав сети, управляемой доменом, весь контроль над защитой компьютера от сетевых атак ложится "на плечи" домена. Поэтому остается только надеяться, что в домене используется соответствующее программное обеспечение, например антивирусная программа.\par\bigskip

\subsection{Выбор сетевого расположения}

Как уже было сказано ранее, Windows 7 позволяет использовать разные варианты сетевого размещения компьютера при работе в составе локальной сети. Выбор или смена сетевого размещения влечет за собой изменения в работе соответствующих механизмов операционной системы.\par\bigskip

Различают следующие варианты сетевого размещения:
\begin{itemize}
\item домашняя сеть. Это сетевое размещение подразумевает, что компьютер входит в состав небольшой локальной сети, участники которой вам знакомы и уровень доверия к которым вполне высокий, что позволяет не беспокоиться о сетевых угрозах. Данный вариант размещения автоматически устанавливается, когда происходит подключение к одной из домашних сетей, например организованных посредством Windows 7.
\item сеть предприятия, или Рабочая сеть. Данное сетевое размещение подразумевает, что компьютер входит в состав рабочей сети, размер которой не столь важен, главное – достаточный уровень доверия, что позволяет оценивать сеть как доверенную.
\item общественная сеть. Это сетевое размещение подразумевает подключение компьютера к случайной или непостоянной сети. Примером такой сети может стать зона Wi-Fi, например, в кафе или аэропорту. По понятным причинам данная сеть обладает наименьшей степенью доверия. При ее использовании активируются соответствующие механизмы защиты операционной системы.
\item доменная сеть. Наиболее доверенный тип сетевого размещения, выбор которого в обычном режиме недоступен. Смена на этот тип сетевого размещения происходит автоматически и только в том случае, когда выполняется подключение компьютера к сети с доменом.
\end{itemize}

Смену сетевого размещения можно производить самостоятельно либо оставить этот выбор на усмотрение операционной системы.\par\bigskip

Если требуется изменить сетевое размещение, воспользуйтесь для этого окном Центр управления сетями и общим доступом, запустить которое можно из Панели управления.

\subsection{Подключение к рабочей группе}

Для начала необходимо открыть механизм Система, запустить который можно с Панели управления. Здесь отображается некоторая информация о компьютере, а также сведения об имени компьютера и его текущей принадлежности к какой-либо сети. Кроме того, здесь находится механизм изменения этого состояния. Чтобы им воспользоваться, перейдите по ссылке Изменить параметры.\par\bigskip

В появившемся окне отображается описание компьютера, его имя и рабочая группа или домен, к которому он принадлежит. Здесь же присутствуют две кнопки, позволяющие подключить компьютер к рабочей группе или домену.\par\bigskip

Для подключения компьютера к рабочей группе щелкните на кнопке Изменить.\par\bigskip

Чтобы подключить компьютер к нужной рабочей группе, достаточно просто ввести ее название в соответствующее поле и нажать кнопку OK. Никакой авторизации при этом не требуется, поскольку сам принцип организации работы рабочей группы подразумевает свободное членство в группе. Буквально через несколько секунд появится окно с подтверждение того, что компьютер подключен к рабочей группе. Вам остается только перезагрузить компьютер, чтобы начать полноценную работу уже в составе этой рабочей группы.

\subsection{Подключение к домену}

Подключение к домену компьютера с операционной системой Windows 7, как и в других операционных системах, требует определенных прав доступа (точнее, прав на подключение компьютера к домену). Кроме того, потребуются данные об учетной записи пользователя, который будет работать на этом компьютере. В этом нет ничего странного, поскольку уровень безопасности в доменной сети подразумевает максимально возможную защиту как вашего компьютера, так и управляющего сервера, который организует работу сети. Процесс подключения к домену, а также добавления сетевого пользователя контролирует мастер подключений.\par\bigskip

Первое, что предстоит сделать, – выбрать направление работы мастера. Мастер универсален: он позволяет подключать компьютер не только к домену, но и к рабочей группе, поэтому, чтобы направить его усилия «в нужное русло», требуется указать соответствующий вариант действий. Выбор очевиден, поэтому, установив переключатель в положение с упоминанием корпоративной сети, продолжаем работу мастера.\par\bigskip

Поскольку наша задача – подключение к домену, выберите соответствующее положение переключателя, в результате чего процесс подключения продолжится.\par\bigskip

Далее мастер вас предупредит, что для подключения к домену нужна определенная информация. В частности, сведения об учетных данных сетевого пользователя, который будет работать на данном компьютере, а также имя компьютера, под которым он будет идентифицирован в сети, если данные о нем не будут найдены в Active Directory. Подготовив эту информацию, продолжите процесс. При этом следует учитывать, что учетная запись пользователя уже должна быть зарегистрирована в Active Directory, иначе подключение будет невозможно.\par\bigskip

Если авторизация будет успешной, появится окно, сообщающее, что для завершения процесса подключения требуется перезагрузка компьютера. В противном случае необходимо будет уточнить данные авторизации либо отказаться от подключения к домену.\par\bigskip

\subsection{Настройка TCP/IP-протокола}

Настройка параметров TCP/IP-протокола требуется в том случае, когда необходимо изменить способ IP-адресации, а также уточнить IP-адреса DNS-серверов и добавить маршруты. Изменить настройки протокола очень просто, и, что самое главное, это можно сделать "на ходу", то есть без перезагрузки компьютера.\par\bigskip

Для выполнения необходимых изменений будем использовать Центр управления сетями и общим доступом, открыть который можно с Панели управления.\par\bigskip

В правой части появившегося окна находится несколько ссылок, позволяющих получить доступ к разным функциям. В частности, чтобы получить доступ к настройкам сетевого адаптера, необходимо использовать ссылку Изменение параметров адаптера. При её нажатии откроется окно, содержащее список всех сетевых подключений, которые используются на компьютере.\par\bigskip

Из всего списка служб и протоколов, которые обслуживают нужное нам сетевое подключение, нас интересует строка Протокол Интернета версии 4 (TCP/IPv4). Дважды щелкните на ней. Появится окно настройки TCP/IP-протокола. Здесь вы можете вносить все необходимые изменения, но главное не ошибиться, поскольку от этого зависит, будет ли компьютер виден в сети.