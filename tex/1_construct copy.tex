\newpage

\section{Конструкторская часть}

\subsection{Разработка технического задания}
\subsubsection{Постановка задачи проектирования}

Целью разработки подсистемы автономного определения перемещения объекта является предоставления удобного с точки зрения интеграции компонента для встраивания во многие бытовые автономные автоматические системы,  в то же время дешевого и не требующего специализированных устройств для своей работы.

\subsubsection{Описание предметной области}
\paragraph{Естественно-языковое описание процесса.}
В процессе функционирования спроектированного модуля происходит следующий бесконечный процесс. 
На вход модуля непрерывно подается видео поток и данные об угловых скоростях и ускорении объекта относительно трех взаимноперпендикулярных осей. Эти данные обрабатываются параллельно в соответствующих модулях, на выходе каждого из которых получаем смещение объекта относительно предыдущего положения и его поворот. Далее эти данные совмещаются и выбираются наиболее правдоподобные, которые затем прибавляются к положению и углу поворота, высчитанным на предыдущей итерации. 


\paragraph{Графическое представление процесса}

\paragraph{Вычисление оптического потока.}
\textbf{Оптический поток} — это изображение видимого движения объектов, поверхностей или краев сцены, получаемое в результате перемещения наблюдателя (глаз или камеры) относительно сцены.

Существует несколько подходов к определению смещений между двумя соседними кадрами. Например, можно для каждого небольшого фрагмента (скажем, 8 на 8 пикселей) одного кадра найти наиболее похожий фрагмент на следующем кадре. В этом случае разность координат исходного и найденного фрагментов даст нам смещение. Основная сложность тут состоит в том, как быстро отыскать нужный фрагмент, не перебирая весь кадр пиксель за пикселем. Различные реализации этого подхода так или иначе решают проблему вычислительной сложности. Некоторые настолько успешно, что применяются, например, в распространенных стандартах сжатия видео. Платой за скорость естественно является качество. Мы же рассмотрим другой подход, который позволяет получить смещения не для фрагментов, а для каждого отдельного пикселя, и применяется тогда, когда скорость не столь критична. Именно с ним в литературе часто связывают термин “оптический поток”.

Данный подход часто называют дифференциальным, поскольку в его основе лежит вычисление частных производных по горизонтальному и вертикальному направлениям изображения. Как мы увидим далее, одних только производных недостаточно чтобы определить смещения. Именно поэтому на базе одной простой идеи появилось великое множество методов, каждый из которых использует какую-нибудь свою математическую пляску с бубном, чтобы достичь цели. Сконцентрируемся на методе Лукаса-Канаде (Lucas-Kanade), предложенном в 81 году Брюсом Лукасом и Такео Канаде.

С математической точки зрения данный алгоритм можно описать следующим образом.
Пусть даны два изображения $F1$ и $F2$, и нам требуется найти смещение точки с координотой $x$. Рассматривая два последовательных изображения можно сказать:
$$ f_2(x) = f_1 (x-d) $$
Обратите внимание, что $f_1$ и $f_2$ при желании можно записать и в общем виде: $f_1(x) = I (x, y, t)$ ; $f_2(x) = I (x, y, t+1)$.

Свяжем известные значения со смещением d. Для этого запишем разложение в ряд Тейлора для $ f_1 (x-d)$:
$$  f_1 (x-d) =f_1(x) + df_1'(x) + O(d^2f_1'') $$
Предположим, что $ f_1 (x-d)$ достаточно хорошо аппроксимируется первой производной. Сделав это предположение, отбросим всё что после первой производной:
$$  f_1 (x-d) =f_1(x) + df_1'(x) $$

Смещение $d$ — это наша искомая величина, поэтому необходимо преобразовать $ f_1 (x-d)$ . Как мы условились ранее, $ f_2(x) = f_1 (x-d) $, поэтому просто перепишем:
$$ f_2(x)= f_1(x) - df_1'(x) $$
Отсюда следует:
$$ d = \frac{f_1(x)-f_2(x)}{f_1'(x)} $$

Следует отметить, что выше был рассмотрен одномерный случай и были сделаны несколько грубых допущений. Но описание алгоритма Лукаса-Канаде для двумерного случая только усложняет математические выводы и понимание сути. 

Для снижения погрешности вызванной отбрасыванием старших производных смещение для каждой пары кадров (назовём их $F_i$ и $F_{i+1}$) можно вычислять итеративно. В литературе это называется искажением (warping). На практике это означает, что, вычислив смещения на первой итерации, мы перемещаем каждый пиксель кадра $F_{i+1}$ в противоположную сторону так, чтобы это смещение компенсировать. На следующей итерации вместо исходного кадра $F_{i+1}$ мы будем использовать его искаженный вариант $F_{i+1}^1$. И так далее, пока на очередной итерации все полученные смещения не окажутся меньше заданного порогового значения. Итоговое смещение для каждого конкретного пикселя мы получаем как сумму его смещений на всех итерациях.

Так же следует отметить, что данный алгоритм плохо работает на однотонных изображениях. Данный недостаток является самым критичным. 

\paragraph{Одометрия с использованием инерциальных измерительных устройств}

\paragraph{Корректировка выходных данных}

\paragraph{Анализ функций, подлежащих автоматизации}

\subsubsection{Выбор критериев качества}

\subsubsection{Анализ аналогов и прототипов}

\paragraph{Сравнительный анализ}

Для сравнения представленных вариантов воспользуемся методом взвешенной суммы. Данный метод позволяет объединить ряд критериев сравнения в один интегральный показатель, по которому затем выбирается наилучший вариант, соответствующий максимальному значению этого интегрального показателя. Метод взвешенной суммы можно представить следующим образом: 
$$ Y = \max_{j \ni m} \displaystyle\sum_{i=1}^{n} \alpha_i \cdot K_{ij},$$
где $\sum_{i=1}^{n} \alpha_i = 1$

По этому критерию проводится сравнение $j (j = 1, 2, …, m)$ вариантов по $i (i = 1, 2, …, n)$ показателям, где:

$n$ – количество показателей сравнения;

$m$ – количество вариантов сравнения.

$K_{ij}$ – нормированный коэффициент соответствия $i$-ого параметра $j$-ого варианта эталонному значению, т.е. для $j$-ого варианта:
$$ K_{ij} = \frac{\max_{j} X_{ij}} {X_{ij}}, $$
$$ 0 < K_{ij} < 1 $$
Соответствие систем-аналогов выбранным критериям качества представлено в Таблице

\subsubsection{Перечень задач, подлежащих решению в процессе разработки}

Исходя из приведенного выше первичного анализа предметной области можно сформировать список задач, подлежащих решению.

Необходимо решить следующие задачи:

\begin{enumerate}
\item разработка структуры и архитектуры подсистемы системы; 
\item разработка требований к формату и структуре передаваемых данных;
\item разработка алгоритмов обработки информации;
\item выбор и обоснование КТС, необходимого для реализации системы;
\item разработка графа диалога и набора экранных форм;
\item оценка предполагаемого качества функционирования системы;
\item организационно-экономическое обоснование разработки;
\item рекомендации по охране труда.
\end{enumerate}

\subsection{Проектирование подсистемы}
\subsubsection{Разработка структуры подсистемы}
\paragraph{Определение состава компонентов}

Исходя из анализа функций структурно в подсистеме можно выделить следующие основные части:
\begin{itemize}
\item \textbf{модуль обработки входных данных} (преобразует входные данные в удобоваримый вариант для последующей обработки);
\item \textbf{модуль компьютерного зрения }(позволяет обрабатывать изображения и производить их анализ для построения визуальной одометрии);
\item \textbf{модуль визуальной одометрии} (высчитывает перемещение и угол поворота камеры на основе последовательности изображений);
\item \textbf{модуль обработки данных с инерционных приборов} (производит математическую обработку показаний датчиков и на ее основе вычисляет перемещение объекта);
\item \textbf{модуль сопоставления  и вывода данных} (сравнивает показания двух предыдущих модулей и на их основе выводит наиболее правдоподобное положение объекта).
\end{itemize}

\paragraph{Определение структуры компонентов}

\paragraph{Описание процессов}

\paragraph{Математическое обеспечение}

\subsubsection{Разработка формата и структуры данных}

\subsubsection{Разработка алгоритмов обработки информации}
\paragraph{Общий алгоритм работы}
\paragraph{Алгоритм вычисления оптического потока}
\paragraph{Алгоритм обработки данных с ИИУ}
\paragraph{Алгоритм сопоставления данных}

