\newpage

\section{Техническое задание}

Исходные данные представлены в таблице~\ref{table:ishodn_dann}.

\begin{table}[ht]
\caption{Исходные данные, 19 вариант}
\label{table:ishodn_dann}
\centering
 \begin{tabular}{|c|c|c|c|c|c|c|}
 \hline Офис & 1 ф & 2 ф & ОС, СУБД и оборудование & ЦП & Диски & Модель \\
 \hline 6(2)/2(2) & 3(4)/2(4) & 14 & 24/34/44 & 54 & 64(3)(0.6) & 74 \\
 \hline
 \end{tabular}
\end{table}

\subsection{Наименование}

Проектирование распределённой АСОИиУ фирмы.

\subsection{Основание для разработки}

Основанием для разработки является учебный план кафедры ИУ5 МГТУ им. Н.Э. Баумана. 

\subsection{Цель разработки}

Целью разработки является создание проектного решения на распределенную сеть обработки информации, объединяющую все подразделения фирмы, состоящей из центрального и двух удаленных офисов.

\subsection{Задачи, подлежащие решению}

В процессе выполнения курсовой работы необходимо решить следующие задачи:
\begin{enumerate}
\item Разработать блок-схему распределенной АСОИиУ фирмы и структурные схемы ЛВС центрального и удаленных офисов фирмы;
\item Описать правила построения всех сетей фирмы;
\item Выбрать и обосновать вариант удаленной связи отдельных ЛВС фирмы;
\item Выбрать требуемое оборудование для ЛВС фирмы;
\item Описать настройку рабочих параметров сетевой ОС, под управлением которой работают ЛВС фирмы;
\item Описать настройку рабочих параметров СУБД, которая установлена в ЛВС фирмы;
\item Провести распределение предметных баз данных по узлам сети;
\item Выполнить аналитическое и имитационное моделирование ЛВС фирмы и провести сравнительный анализ результатов моделирования;
\item Разработать и представить рекомендации по модернизации и реорганизации распределенной АСОИиУ фирмы.
\end{enumerate}

\subsection{Требования к составу технических средств}

Фирма состоит из центрального офиса и двух удалённых филиалов.\par\bigskip

В центральном офисе фирмы расположены ЛВС 100BASE-T4, содержащая 2 концентратора, и ЛВС 10BASE-2, содержащая 2 сегмента. Обе сети подключены к коммутатору, к которому также подключён удаленный маршрутизатор с двумя модемами.\par\bigskip

В первом филиале фирмы расположены ЛВС 10BASE-T, содержащая 4 концентратора, и ЛВС 10BASE-2, содержащая 4 сегмента. Обе сети подключены к коммутатору, к нему также подключен удаленный маршрутизатор с одним модемом.\par\bigskip

Во втором филиале фирмы расположена ЛВС Token Ring на STP c усилителями.\par\bigskip

В сети установлен сервер на базе 4 ядерного процессора с дисковой подсистемой уровня RAID-6.
 
\subsection{Требования к составу программных средств}

ЛВС работают под управлением сетевой ОС Windows 7. В сети установлена СУБД Sybase.

\subsection{Техническая документация, предъявляемая по окончании работы}

По окончании работы должна быть предъявлена следующая документация:
\begin{itemize}
\item структурная схема объединённой сети фирмы;
\item распределение предметных баз данных по узлам сети;
\item формализованная схема сети и результаты моделирования;
\item пояснительная записка.
\end{itemize}

\subsection{Порядок приема работы}

Приём работы осуществляется путем проверки соответствия выполненной работы пунктам технического задания.

\subsection{Дополнительные условия}

Определить вероятность безотказной работы дисковой подсистемы сервера, построенной на базе RAID-6, содержащей 3 базовых диска (без учета уровня RAID), при условии, что вероятность безотказной работы одного диска равна 0.6 и все диски одинаковые.\par\bigskip
 
Необходимо провести сравнительный анализ серверов отдела на базе 4 ядерных процессоров и выбрать наилучший. Подробно описать особенности эксплуатации блока питания и ИБП.\par\bigskip

Модель системы должна соответствовать общему виду, чтобы можно было провести исследование любого варианта. Результаты моделирования должны соответствовать варианту задания, по-этому необходимо провести моделирование системы, содержащей 19 рабочих станций и сервер (ЦП и диски).