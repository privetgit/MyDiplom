\newpage

\section{Настройка рабочих параметров СУБД}

Согласно ТЗ, в качестве СУБД используется СУБД Sybase от одноимённой компании.

\subsection{Инсталляция}

На сервере должно быть установлено сетевое программное обеспечение: сетевой протокол (например, TCP/IP, NetBEUI или SPX/IPX), клиентская часть для работы в сети (Client for Microsoft Networks) для машин на Intel платформе. Для использования протокола TCP/IP серверу должен быть присвоен уникальный IP-адрес. Проверка правильности установки протокола TCP/IP может быть произведена с помощью команды: \verb|ping <servername>|, запущенной с любой другой машины в сети. При правильной настройке будет получен ответ: \verb|Replay from ...|\par\bigskip

Сначала создается глобальная переменная среды SYBASE, которой присваивается значение каталога, в который производится установка, например:

\verb|D:\SYBASE|.\par\bigskip

Запуск инсталляции производится с помощью команды setup.exe с CD-диска, если он не начался автоматически после вставки инсталляционного диска в CD-ROM.\par\bigskip

Важным моментом при инсталляции является выбор кодовой страницы и порядка сортировки, а также номер порта, по которому будет осуществляться взаимодействие клиентских приложений с ASE по протоколу TCP/IP. Номер порта может быть изменен впоследствии только при переустановке Sybase ASE. Рекомендуемые значения параметров для кодовой страницы и порядка сортировки: Character Set - 1251 (id=53), Sort Order - Case Insensitive, Accent Sensitive (id=59).\par\bigskip

\subsection{Установка параметров и настройка}

Полный список всех параметров вызывается с помощью системной процедуры \verb|sp_configure| и хранится в системной таблице \verb|sysconfigures|. Часть параметров после изменения не требует перезапуска сервера, так как являются динамическими. Чтобы проверить, требуется ли перезапуск сервера после изменения значения параметра, нужно сравнить значения \verb|config_value| и \verb|run_value|, возвращаемые командой:

\verb|sp_configure <имя_параметра>|\par\bigskip

При равенстве этих значений перезапуск не требуется. Для изменения значения параметра используется команда:

\verb|sp_configure <имя_параметра>, <новое_значение>|\par\bigskip

Далее приводятся основные параметры настройки.\par\bigskip

\verb|Allow updates to system tables = 0|

Запрещено обновление системных таблиц. Выполняется для предотвращения случайного изменения таблиц пользователями. Параметр может быть временно сделан равным 1 при необходимости изменения системных таблиц администратором системы, после чего снова обязательно должен быть обнулен.\par\bigskip

\verb|Deadlock checking period = 5000|

Время в миллисекундах, после прохождения которого ожидающий освобождения ресурса процесс признается deadlock'ом. Увеличение значения параметра по сравнению со значением по умолчанию позволяет сократить количество проверок deadlock'ов в единицу времени и избежать ошибочных присвоений ожидающим процессам статуса deadlock.\par\bigskip

\verb|Default character set id = 53|

Устанавливается кодовая страница cp1251, поддерживающая русский язык. Значение параметра устанавливается в процессе инсталляции сервера, и при изменении его с помощью команды \verb|sp_configure| требуется либо предварительная выгрузка и повторная загрузка всех баз данных из-за изменения формата хранения данных, либо проведение ряда административных мероприятий после автоматической конвертации баз.\par\bigskip

\verb|Default sort order id = 59|

По аналогии с предыдущим параметром устанавливается алфавитный регистронезависимый порядок сортировки. Оба эти параметра рекомендуется устанавливать сразу при инсталляции системы до создания рабочих баз данных.\par\bigskip

\verb|Lock scheme = datapages|

Устанавливает тип блокировки для создаваемых таблиц по умолчанию в значение 'datapages' (постраничная блокировка).\par\bigskip

\verb|Max online engines = n|

Устанавливается количество логических процессоров, которое будет предоставлено Sybase для обработки запросов.\par\bigskip

\verb|Number of devices = n|

По умолчанию значение этого параметра = 10. Если на сервере необходимо создать больше устройств баз данных, то это можно сделать, соответственно увеличив значение параметра.\par\bigskip

\verb|Number of locks = 50000|

Количество блокировок, которые одновременно могут быть установлены процессами всех работающих пользователей.\par\bigskip

\verb|Number of open databases = n|

Устанавливается по количеству реально существующих баз данных на сервере, включая системные и пользовательские базы.\par\bigskip

\verb|Number of open indexes = 2009|

Количество одновременно открытых индексов.\par\bigskip

\verb|Number of open objects = 50000|

Количество одновременно открытых объектов баз данных.\par\bigskip

\verb|Number of user connections = users|

Значение параметра должно соответствовать увеличенному в пять раз количеству пользователей, которые могут одновременно работать на сервере.\par\bigskip

\verb|Page lock promotion HWM = 20000|

Пороговое значение, определяющее, после какого количества страничных блокировок в таблице к ней будет применена полная блокировка.\par\bigskip

\verb|Row lock promotion HWM = 20000|

Пороговое значение, определяющее, после какого количества позаписных блокировок в таблице к ней будет применена полная блокировка.\par\bigskip

\verb|User log cache size = 4096|

Размер пользовательского буфера транзакций, в который накапливаются транзакции перед записью в transaction log. Увеличение этого параметра позволяет уменьшить количество конфликтов, возникающих при одновременной попытке нескольких процессов выполнить запись в transaction log.\par\bigskip

Кроме того, в файле \verb|%SYBASE%\locales\locales.dat| нужно для секции \verb|[nt]| заменить строки:
\begin{verbatim}
locale = enu, us_english, iso_1
locale = default, us_english, iso_1
\end{verbatim}
на:
\begin{verbatim}
locale = enu, us_english, cp1251
locale = default, us_english, cp1251
\end{verbatim}