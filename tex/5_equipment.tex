\newpage

\section{Организация связи с филиалами}

\subsection{Выбор технологии}

Мною было рассмотрено три варианта организации удалённых связей сети фирмы: X.25, Frame Relay и ADSL.\par\bigskip

Стандарт X.25, как правило используется для организации международных сетей. Для связи локальной сети с сетью X.25 используется мост или маршрутизатор. Доступ к сети осуществляется через арендуемую линию или линию с вызовом по номеру. В выделенных линиях обычно используют синхронную связь, что увеличивает пропускную способность. Скорость передачи составляет 19.2 - 64 Кбит/c. Линии с вызовом по номеру используют асинхронные методы с применением модемов, которые имеют собственные средства коррекции ошибок. Скорость передачи зависит от скорости модема.\par\bigskip

Поскольку стандарт X.25 предусмотрен для организации международных сетей, то и стоимость использования этой сети достаточно высока. При скорости 64 Кбит/c стоимость подключения составляет 30000 рублей, а абонентская плата - 4200 рублей в месяц.\par\bigskip

Сети Frame Relay - сравнительно новые сети, которые гораздо лучше подходят для передачи пульсирующего трафика локальных сетей по сравнению с сетями Х.25. Преимущество сетей Frame Relay заключается в их низкой протокольной избыточности и дейтаграммном режиме работы, что обеспечивает высокую пропускную способность и небольшие задержки кадров. Надежную передачу кадров технология Frame Relay не обеспечивает. Сети Frame Relay специально разрабатывались как общественные сети для соединения частных локальных сетей. Они обеспечивают скорость передачи данных до 2 Мбит/с.\par\bigskip

Услуги Frame Relay обычно предоставляются теми же провайдерами, которые эксплуатируют сети Х.25. Большая часть производителей выпускают сейчас коммутаторы, которые могут работать как по протоколам Х.25, так и по протоколам Frame Relay. Технология Frame Relay начинает занимать в территориальных сетях с коммутацией пакетов ту же нишу, которая заняла в локальных сетях технология Ethernet.\par\bigskip

Сети Frame Relay следует применять только при наличии на магистральных каналах волоконно-оптических кабелей высокого качества. Стоимость использования данной сети достаточно высока. Подключение стоит 57000 рублей и абонентская плата составляет 9000 рублей в месяц.\par\bigskip

ADSL модемы передают данные намного быстрее, чем обычные аналоговые модемы, используя те же телефонные линии. ADSL обеспечивает высокоскоростной широковещательный доступ. Скорость передачи данных достигает 8 Мб/с. ADSL работает по имеющейся телефонной линии и при этом не занимает телефонный канал. Таким образом, возможно одновременно пользоваться телефонным аппаратом и осуществлять доступ в интернет. Фактически, ADSL модем образует три канала:
\begin{itemize}
\item входящий канал, скорость до 8 Мбит/с, 
\item исходящий канал, скорость до 1 Мб/с, 
\item обычный канал телефонной связи, по которому ведутся телефонные разговоры.
\end{itemize}\par\bigskip

Скорость передачи данных зависит от используемого оборудования, длины и качества телефонной линии. ADSL не требует, как аналоговый модем, набора номера для установления соединения с сетью. Обычно, телефонные сервисы используют минимальную часть пропускной способности телефонной линии, ADSL занимает оставшуюся часть для реализации высокоскоростной передачи данных. Подключение стоит 3000 рублей, а абонентская плата составляет 1500 рублей в месяц.\par\bigskip

Качественные характеристики рассматриваемых технологий сведены в таблицу~\ref{table:ISP_compare_qual}.

\begin{table}[h]
\caption{Качественные характеристики технологий передачи данных}
\label{table:ISP_compare_qual}
\centering
  \begin{tabular}{|c|c|c|c|}
  \hline Параметр & X.25 & Frame Relay & ADSL \\
  \hline Скорость передачи, Кбит/с & 64 & 2000 & 8000 \\
  \hline Надёжность передачи данных & очень хорошая & отличная & хорошая \\
  \hline Возможность масштабирования & отличная & отличная & плохая \\
  \hline Стоимость подключения, рублей & 30000 & 57000 & 3000 \\
  \hline Абонентская плата, рублей & 4200 & 9000 & 1500 \\
  \hline
  \end{tabular}
\end{table}

Количественные характеристики рассматриваемых технологий сведены в таблицу~\ref{table:ISP_compare_numb}.

\begin{table}[h]
\caption{Количественные характеристики технологий передачи данных}
\label{table:ISP_compare_numb}
\centering
  \begin{tabular}{|c|c|c|c|}
  \hline Параметр & X.25 & Frame Relay & ADSL \\
  \hline Скорость передачи, Кбит/с & 64 & 2000 & 8000 \\
  \hline Надёжность передачи данных & 5 & 6 & 4 \\
  \hline Возможность масштабирования & 6 & 6 & 1 \\
  \hline Стоимость подключения, рублей & 30000 & 57000 & 3000 \\
  \hline Абонентская плата, рублей & 4200 & 9000 & 1500 \\
  \hline
  \end{tabular}
\end{table}

Теперь пронормируем параметры и добавим весовые коэффициенты. Результаты приведены в таблице~\ref{table:ISP_compare}.

\begin{table}[h]
\caption{Нормированные параметры рассматриваемых технологий}
\label{table:ISP_compare}
\centering
  \begin{tabular}{|c|c|c|c|c|}
  \hline Параметр & Коэфф. & X.25 & Frame Relay & ADSL \\
  \hline Скорость передачи, Кбит/с & 0.2 & 0.008 & 0.25 & 1 \\
  \hline Надёжность передачи данных & 0.3 & 0.833 & 1 & 0.666 \\
  \hline Возможность масштабирования & 0.2 & 1 & 1 & 0.166 \\
  \hline Стоимость подключения, рублей & 0.1 & 0.1 & 0.053 & 1 \\
  \hline Абонентская плата, рублей & 0.2 & 0.357 & 0.166 & 1 \\
  \hline Итог & 1 & 0.5329 & 0.5885 & 0.733 \\
  \hline
  \end{tabular}
\end{table}

Таким образом, выбор падает на технологию ADSL.

\subsection{Выбор оборудования}

\subsubsection{Выбор маршрутизатора}

Выбираем из трёх маршрутизаторов:
\begin{enumerate}
\item Asus RT-G32;
\item Cisco 2911R;
\item D-Link DIR-300.
\end{enumerate}

Качественные характеристики рассматриваемых маршрутизаторов сведены в таблицу~\ref{table:router_compare_qual}.

\begin{table}[h]
\caption{Качественные характеристики маршрутизаторов}
\label{table:router_compare_qual}
\centering
  \begin{tabular}{|c|c|c|c|}
  \hline Параметр & 1 & 2 & 3 \\
  \hline Количество портов & 4 & 8 & 12 \\
  \hline Сложность настройки & средняя & высокая & низкая \\
  \hline Размер таблицы маршрутизации & 10 & 20 & 25 \\
  \hline Гарантия, месяцев & 6 & 24 & 12 \\
  \hline Стоимость, рублей & 5000 & 8000 & 2000 \\
  \hline
  \end{tabular}
\end{table}

Количественные характеристики рассматриваемых маршрутизаторов сведены в таблицу~\ref{table:router_compare_numb}.

\begin{table}[h]
\caption{Количественные характеристики маршрутизаторов}
\label{table:router_compare_numb}
\centering
  \begin{tabular}{|c|c|c|c|}
  \hline Параметр & 1 & 2 & 3 \\
  \hline Количество портов & 4 & 8 & 12 \\
  \hline Сложность настройки & 2 & 1 & 3 \\
  \hline Размер таблицы маршрутизации & 10 & 25 & 20 \\
  \hline Гарантия, месяцев & 6 & 24 & 12 \\
  \hline Стоимость, рублей & 5000 & 8000 & 2000 \\
  \hline
  \end{tabular}
\end{table}

Теперь пронормируем параметры и добавим весовые коэффициенты. Результаты приведены в таблице~\ref{table:router_compare}.

\begin{table}[h]
\caption{Нормированные параметры маршрутизаторов}
\label{table:router_compare}
\centering
  \begin{tabular}{|c|c|c|c|c|}
  \hline Параметр & Коэфф. & 1 & 2 & 3 \\
  \hline Количество портов & 0.3 & 0.333 & 0.666 & 1 \\
  \hline Сложность настройки & 0.1 & 0.666 & 0.333 & 1 \\
  \hline Размер таблицы маршрутизации & 0.2 & 0.4 & 1 & 0.8 \\
  \hline Гарантия, месяцев & 0.2 & 0.25 & 1 & 0.5 \\
  \hline Стоимость, рублей & 0.2 & 0.4 & 0.25 & 1 \\
  \hline Итог & 1 & 0.3765 & 0.6831 & 0.86 \\
  \hline
  \end{tabular}
\end{table}

Таким образом, выбираем маршрутизатор D-Link DIR-300.

\subsubsection{Выбор модема}

Выбираем из трёх модемов:
\begin{enumerate}
\item D-Link DSL-2500U;
\item Asus DSL-N10;
\item Trendnet TDM-C504.
\end{enumerate}

Качественные характеристики рассматриваемых модемов сведены в таблицу~\ref{table:modem_compare_qual}.

\begin{table}[h]
\caption{Качественные характеристики модемов}
\label{table:modem_compare_qual}
\centering
  \begin{tabular}{|c|c|c|c|}
  \hline Параметр & 1 & 2 & 3 \\
  \hline Стандарты ADSL & ADSL, ADSL2, ADSL2+ & ADSL, ADSL2 & ADSL \\
  \hline Стандарты VPN & IPSec, PL2TP & IPSec, PPTP, PL2TP & IPSec \\
  \hline Flash-память, МБ & 4 & 8 & 20 \\
  \hline Гарантия, месяцев & 6 & 24 & 12 \\
  \hline Стоимость, рублей & 2000 & 3000 & 1500 \\
  \hline
  \end{tabular}
\end{table}

Количественные характеристики рассматриваемых модемов сведены в таблицу~\ref{table:modem_compare_numb}.

\begin{table}[h]
\caption{Количественные характеристики модемов}
\label{table:modem_compare_numb}
\centering
  \begin{tabular}{|c|c|c|c|}
  \hline Параметр & 1 & 2 & 3 \\
  \hline Стандарты ADSL & 3 & 2 & 1 \\
  \hline Стандарты VPN & 2 & 3 & 1 \\
  \hline Flash-память, МБ & 4 & 8 & 20 \\
  \hline Гарантия, месяцев & 6 & 24 & 12 \\
  \hline Стоимость, рублей & 2000 & 3000 & 1500 \\
  \hline
  \end{tabular}
\end{table}

Теперь пронормируем параметры и добавим весовые коэффициенты. Результаты приведены в таблице~\ref{table:modem_compare}.

\begin{table}[h]
\caption{Нормированные параметры модемов}
\label{table:modem_compare}
\centering
  \begin{tabular}{|c|c|c|c|c|}
  \hline Параметр          & Коэфф. & 1      & 2      & 3 \\
  \hline Стандарты ADSL    & 0.3    & 1      & 0.5    & 0.333 \\
  \hline Стандарты VPN     & 0.3    & 0.5    & 1      & 0.333 \\
  \hline Flash-память, МБ  & 0.1    & 0.2    & 0.4    & 1 \\
  \hline Гарантия, месяцев & 0.2    & 0.25   & 1      & 0.5 \\
  \hline Стоимость, рублей & 0.1    & 0.75   & 0.5    & 1 \\
  \hline Итог              & 1      & 0.545  & 0.74   & 0.4998 \\
  \hline
  \end{tabular}
\end{table}

Таким образом, выбираем модем Asus DSL-N10.

\subsubsection{Выбор сервера}

Выбираем из трёх серверов на базе 4 ядерных процессоров:
\begin{enumerate}
\item ASUS RS720-E6-R;
\item Dell E3-1240V2;
\item IBM KVR16E11/4.
\end{enumerate}

Выбор весовых коэффициентов параметров сравнения серверов будет проводиться с использованием метода анализа иерархий, а выбор сервера - с использованием метода взвешенной суммы параметров сравнения.

Качественные характеристики рассматриваемых серверов сведены в таблицу~\ref{table:server_compare_qual}.

\begin{table}[h]
\caption{Качественные характеристики серверов}
\label{table:server_compare_qual}
\centering
  \begin{tabular}{|c|c|c|c|}
  \hline Параметр & 1 & 2 & 3 \\
  \hline Частота процессора, ГГц & 2.4 & 3.4 & 3.1 \\
  \hline Оперативная память, ГБ & 12 & 10 & 16 \\
  \hline Жёсткий диск, ТБ & 2 & 4 & 1 \\
  \hline Блок питания, Вт & 500 & 700 & 650 \\
  \hline Горячая замена дисков & легко & непросто & затруднительно \\
  \hline Сетевой адаптер & LAN, WiFi & LAN & LAN, WiFi\\
  \hline Стоимость, рублей & 79000 & 39000 & 43000 \\
  \hline
  \end{tabular}
\end{table}

Количественные характеристики рассматриваемых серверов сведены в таблицу~\ref{table:server_compare_numb}.

\begin{table}[h]
\caption{Количественные характеристики серверов}
\label{table:server_compare_numb}
\centering
  \begin{tabular}{|c|c|c|c|}
  \hline Параметр                & 1     & 2     & 3 \\
  \hline Частота процессора, ГГц & 2.4   & 3.4   & 3.1 \\
  \hline Оперативная память, ГБ  & 12    & 10    & 16 \\
  \hline Жёсткий диск, ТБ        & 2     & 4     & 1 \\
  \hline Блок питания, Вт        & 500   & 700   & 650 \\
  \hline Горячая замена дисков   & 4     & 3     & 2 \\
  \hline Сетевой адаптер         & 3     & 2     & 3 \\
  \hline Стоимость, рублей       & 79000 & 39000 & 43000 \\
  \hline
  \end{tabular}
\end{table}

Весовые коэффициенты определяем методом анализа иерархий. Вычисление осуществляется посредством электронных таблиц MS Excel, которые находятся в приложении. Полученные результаты приведены в таблице~\ref{table:server_compare_veskoef}.

\begin{table}[h]
\caption{Весовые коэффициенты}
\label{table:server_compare_veskoef}
\centering
  \begin{tabular}{|c|c|c|c|c|c|c|c|c|c|}
  \hline    & К1    & К2    & К3    & К4  & К5 & К6    & К7   & Вектор & $\alpha$ \\
  \hline К1 & 1     & 1     & 2     & 4   & 6  & 4     & 5    & 2.67   & 0.284 \\
  \hline К2 & 1     & 1     & 2     & 4   & 6  & 4     & 5    & 2.67   & 0.284 \\
  \hline К3 & 0.5   & 0.5   & 1     & 3   & 5  & 4     & 4    & 1.79   & 0.192 \\
  \hline К4 & 0.25  & 0.25  & 0.333 & 1   & 2  & 1     & 0.5  & 0.57   & 0.061 \\
  \hline К5 & 0.167 & 0.167 & 0.2   & 0.5 & 1  & 0.333 & 0.25 & 0.3    & 0.032 \\
  \hline К6 & 0.25  & 0.25  & 0.25  & 1   & 3  & 1     & 0.5  & 0.59   & 0.062 \\
  \hline К7 & 0.2   & 0.2   & 0.25  & 2   & 4  & 2     & 1    & 0.77   & 0.082 \\
  \hline
  \end{tabular}
\end{table}

Проводим расчёт оценки согласованности. Для $m_f = 7$, $R = 1.32$:

$$\lambda_{max} = \sum_{i=1}^{m_f}\Bigr(a_i\cdot\sum_{j=1}^{m_f}x_{ij}\Bigl)=7.2566$$

$$\text{ОцСогл} = \frac{\lambda_{max} - m_f}{(m_f - 1)\cdot R} = \frac{8.5594 - 7}{(7 - 1)\cdot 1.32} = 0.056$$

Как видим, $0.056 < 0.1$, значит матрица согласована.\par\bigskip

Теперь пронормируем параметры и добавим весовые коэффициенты. Результаты приведены в таблице~\ref{table:server_compare}.

\begin{table}[h]
\caption{Нормированные параметры серверов}
\label{table:server_compare}
\centering
  \begin{tabular}{|c|c|c|c|c|}
  \hline Параметр                & Коэфф. & 1     & 2     & 3 \\
  \hline Частота процессора, ГГц & 0.284  & 0.706 & 1     & 0.911 \\
  \hline Оперативная память, ГБ  & 0.284  & 0.75  & 0.625 & 1 \\
  \hline Жёсткий диск, ТБ        & 0.192  & 0.5   & 1     & 0.25 \\
  \hline Блок питания, Вт        & 0.061  & 0.714 & 1     & 0.929 \\
  \hline Горячая замена дисков   & 0.032  & 1     & 0.75  & 0.5 \\
  \hline Сетевой адаптер         & 0.062  & 1     & 0.666 & 1 \\
  \hline Стоимость, рублей       & 0.082  & 0.494 & 1     & 0.907 \\
  \hline Итог                    & 1      & 0.59356 & 0.86179 & 0.79978 \\
  \hline
  \end{tabular}
\end{table}

Таким образом, выбираем сервер Dell E3-1240V2.

\subsubsection{Выбор ИБП для сервера}

Выбираем из трёх ИБП:
\begin{enumerate}
\item SVEN Power Pro+ 500 ;
\item PowerCom Warrior WAR-500A ;
\item Ippon Back Verso 400.
\end{enumerate}

Качественные характеристики рассматриваемых ИБП сведены в таблицу~\ref{table:ibp_compare_qual}.

\begin{table}[h]
\caption{Качественные характеристики ИБП}
\label{table:ibp_compare_qual}
\centering
  \begin{tabular}{|c|c|c|c|}
  \hline Параметр & 1 & 2 & 3 \\
  \hline Размеры & большой & средний & средний \\
  \hline Время реакции, мс & 10 & 4 & 6 \\
  \hline Автономная работа, мин & 8 & 10 & 13 \\
  \hline Время зарядки & 8 & 6 & 8 \\
  \hline Стоимость, рублей & 1500 & 1400 & 1700 \\
  \hline
  \end{tabular}
\end{table}

Количественные характеристики рассматриваемых ИБП сведены в таблицу~\ref{table:ibp_compare_numb}.

\begin{table}[h]
\caption{Количественные характеристики ИБП}
\label{table:ibp_compare_numb}
\centering
  \begin{tabular}{|c|c|c|c|}
  \hline Параметр & 1 & 2 & 3 \\
  \hline Размеры & 1 & 2 & 2 \\
  \hline Время реакции, мс & 10 & 4 & 6 \\
  \hline Автономная работа, мин & 8 & 10 & 13 \\
  \hline Время зарядки & 8 & 6 & 8 \\
  \hline Стоимость, рублей & 1500 & 1400 & 1700 \\
  \hline
  \end{tabular}
\end{table}

\newpage

Теперь пронормируем параметры и добавим весовые коэффициенты. Результаты приведены в таблице~\ref{table:ibp_compare}.

\begin{table}[h]
\caption{Нормированные параметры ИБП}
\label{table:ibp_compare}
\centering
  \begin{tabular}{|c|c|c|c|c|}
  \hline Параметр               & Коэфф. & 1      & 2      & 3 \\
  \hline Размеры                & 0.1    & 0.5    & 1      & 1 \\
  \hline Время реакции, мс      & 0.3    & 0.4    & 1      & 0.666 \\
  \hline Автономная работа, мин & 0.3    & 0.615  & 0.769  & 1 \\
  \hline Время зарядки, ч       & 0.2    & 0.75   & 1      & 0.75 \\
  \hline Стоимость, рублей      & 0.1    & 0.933  & 1      & 0.824 \\
  \hline Итог                   & 1      & 0.5978 & 0.9307 & 0.8322 \\
  \hline
  \end{tabular}
\end{table}

Таким образом, выбираем ИБП PowerCom Warrior WAR-500A.