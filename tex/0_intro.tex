\section*{Введение}
В ходе создания подвижных автономных систем возникает задача организации ее навигации в пространстве. Данную задачу можно разделить на две:

\begin{itemize}
\item определение текущего местоположения объекта в пространстве;
\item прокладывание дальнейшего маршрута к точке назначения.
\end{itemize}
Данная работа посвещена решению первой задачи, так как она явлется первоочередной в построении системы навигации автономных систем.

Для определения текущего положения объекта можно использовать различные методы, основанные на глобальном позиционировании в географической системе координат с использованием позиционирования по спутникам (GPS, ГЛОНАСС), или методы, основанные на определении перемещения от стартовой позиции. При этом у данных методов разные сферы применения. Так, например, при позиционировании внутри помещения использование спутниковых систем позиционирвоания становиться невозможным по причине слабого сигнала или его полного отсутсвтвия, а так же из-за недостаточной точности в рамках навигации внутри интерьера помещения. Так же не стоит забывать про акутальность систем навигации в космической отрасли, где использование спутников является невозможным впринципе. 
Вторую группу методов принято называть методами одометрии, которые могут быть основаны на:
\begin{itemize}
\item на вращении колес;
\item использовании инерциальных измерительных приборов;
\item компьютерном зрении.
\end{itemize}

Каждый из них обладает своими плюсами и минусами \cite{odometryMethods}, но развитие вычислительной техники и алгоритмов компьютерного зрения дало мощный толчок к более широкому применению визуальной одометрии. Данный подход позволяет получать один видеоряд через видеокамеру и на его основе получать разные сведения об окружающей среде. Тем не менее он не лишен недостатков. Для борьбы с ними применяется комбинация нескольких методов одомтерии. 

