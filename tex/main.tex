\documentclass[a4paper,14pt]{report} %размер бумаги устанавливаем А4, шрифт 12пунктов
\usepackage[T2A]{fontenc}
\usepackage[utf8]{inputenc}%включаем свою кодировку: koi8-r или utf8 в UNIX, cp1251 в Windows
\usepackage[english,russian]{babel}%используем русский и английский языки с переносами
\usepackage{amssymb,amsfonts,amsmath,mathtext,cite,enumerate,float} %подключаем нужные пакеты расширений
\usepackage[dvips]{graphicx} %хотим вставлять в диплом рисунки?
\graphicspath{{images/}}%путь к рисункам

\makeatletter
\renewcommand{\@biblabel}[1]{#1.} % Заменяем библиографию с квадратных скобок на точку:
\makeatother

\usepackage{geometry} % Меняем поля страницы
\geometry{left=2cm}% левое поле
\geometry{right=1.5cm}% правое поле
\geometry{top=1cm}% верхнее поле
\geometry{bottom=2cm}% нижнее поле

\renewcommand{\theenumi}{\arabic{enumi}}% Меняем везде перечисления на цифра.цифра
\renewcommand{\labelenumi}{\arabic{enumi}}% Меняем везде перечисления на цифра.цифра
\renewcommand{\theenumii}{.\arabic{enumii}}% Меняем везде перечисления на цифра.цифра
\renewcommand{\labelenumii}{\arabic{enumi}.\arabic{enumii}.}% Меняем везде перечисления на цифра.цифра
\renewcommand{\theenumiii}{.\arabic{enumiii}}% Меняем везде перечисления на цифра.цифра
\renewcommand{\labelenumiii}{\arabic{enumi}.\arabic{enumii}.\arabic{enumiii}.}% Меняем везде перечисления на цифра.цифра
\renewcommand{\thechapter}{\arabic{chapter}.}\renewcommand{\thesubsection}{\arabic{section}.\arabic{subsection}.}




\begin{document}
\begin{titlepage}
\newpage

\begin{center}
Титульный лист
\end{center}



\end{titlepage}% это титульный лист
\tableofcontents % это оглавление, которое генерируется автоматически



\chapter{Экономическое обоснование разработки и использования ПАОПО}


Обоснование сметы  затрат на разработку программного продукта ПАОПО

Процесс разработки сложного программного продукта сопровождается необходимостью решения многих экономических проблем. Одна из важных экономических проблем – определение стоимости программного продукта (ПП), т.е.  сметной стоимости затрат  на его разработку.

Затраты на разработку программного продукта могут быть представлены в виде сметы затрат, включающей в себя следующие статьи:
\begin{itemize}
	\item расходные материалы;
	\item затраты на оборудование;
	\item затраты на оплату труда;
	\item накладные расходы;
	\item услуги сторонних организаций;
	\item прочие расходы;
\end{itemize}


\newpage

\begin{center}
Экономическое обоснование разработки и использования ПАОПО
\end{center}

Обоснование сметы  затрат на разработку программного продукта ПАОПО

Процесс разработки сложного программного продукта сопровождается необходимостью решения многих экономических проблем. Одна из важных экономических проблем – определение стоимости программного продукта (ПП), т.е.  сметной стоимости затрат  на его разработку.

Затраты на разработку программного продукта могут быть представлены в виде сметы затрат, включающей в себя следующие статьи:
•	расходные материалы;
•	затраты на оборудование;
•	затраты на оплату труда;
•	накладные расходы;
•	услуги сторонних организаций;
•	прочие расходы;


\end{document}